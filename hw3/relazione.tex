\documentclass{article}
\usepackage[english]{babel}
\usepackage[utf8x]{inputenc}
\usepackage{mathtools}
\usepackage{graphicx}
\graphicspath{{images/}}
\usepackage{amsfonts}
\usepackage{amsmath}
\usepackage{subcaption}


\title{Homework 3}
\author{Alessandro Nucci}
\date{June 2019}

\begin{document}

\maketitle

\section{Introduction}
The system that we are going to study is composed of N = 70 monoatomic
molecules in a cubic box of length L and volume $V = L^3$, with periodical
boundary conditions, interacting through a pair potential of the
following type
\begin{equation}
	V(r) = A\sigma^2\frac{e^{−r/\sigma}}{r^2}+ Be^{−2r/\sigma}
\end{equation}
Which bring us to a potential energy defined as
$$\begin{cases}
	U(r) = V(r) − V(r_c) − (r − r_c)V'(r_c) \quad for \: r < r_c\\
	U(r) = 0 \quad for \: r \geq r_c
\end{cases}$$
B was chosen to be $A/2$, $r_c$ was set to $L/2$, $\rho/\sigma^3$ was
fixed at 0.7, and the calculations have been carried in reduced units.\\
The starting configuration of positions and velocities was generated
sampling a uniform distribution, then the velocities sum was set to 0
and the kinetic energy per particle was fixed to 1. Then the same
configuration has been evolved for 1 units with a time step of 0.002
and the kinetic energy has been rescaled to the same value as before.\\
The five simulations done have been evolved for 30 time units with the
following time steps
\begin{center}
\begin{tabular}{c|c|c|c|c}
	0.002 & 0.006 & 0.018 & 0.054 & 0.162
\end{tabular}
\end{center}

\section{Stable runs}
The first three time steps gave stable results, as shown later, while
the last two had a divergence in energy as we can see right under
\begin{center}
	\includegraphics[width=\linewidth]{plot/dump1.png}
	\includegraphics[width=\linewidth]{plot/dump2.png}
\end{center}

\section{Trajectory divergence}
Here are shown the divergencies of the trajectories for energy,
temperature and pressure, i.e. the difference at any time between an
observable computed by the most stable simulation (the one with smallest
time step), and the same observable for all the other stable runs

\begin{center}
	\includegraphics[width=\linewidth]{plot/divE.png}
	\includegraphics[width=\linewidth]{plot/divT.png}
	\includegraphics[width=\linewidth]{plot/divP.png}
\end{center}

\section{Energy conservation}
From the plot below we can see that energy is approximately conserved
with an error growing with the time step
\begin{center}
	\includegraphics[width=\linewidth]{plot/plotE.png}
	\includegraphics[width=\linewidth]{plot/plotE2.png}
	\includegraphics[width=\linewidth]{plot/plotE3.png}
\end{center}

\section{Thermalization}
From the subsequent plots, relatives to pressure and temperature,
$t_{eq}$ is evaluated to 0.3 time units (the plots were zoomed to make
thermalization noticeable)
\begin{center}
	\includegraphics[width=\linewidth]{plot/P.png}
	\includegraphics[width=\linewidth]{plot/T.png}
\end{center}

\section{Errors and correlations}
In conclusion, here are the averages, integrated autocorrelation times
and computed errors for potential energy, pressure and temperature. All
of these measures are consistent within the considered time steps

\begin{center}
\begin{tabular}{c|ccc}
	& U & P & T\\
	\hline
	average & 98.85 & 0.55 &0.7879 \\
	$\tau$ & 0.13 & 0.13 &0.13 \\
	error & 0.4 & 0.0003 &0.0004
\end{tabular}
\end{center}

\end{document}
